
%%%%%%%%%%%%%%%%%%%%%%%Opsætning af format%%%%%%%%%%%%%%%%%%%%%%%%%%
%\documentclass[a4paper,oneside]{memoir} %A4papir, en side, størrelse 12, type memoir 		

%\usepackage[danish,english]{babel}				% dansk og engelsk opsætning
\usepackage[english]{babel}				% dansk og engelsk opsætning
%\renewcommand{\danishhyphenmins}{22}			% fikser babel fejl/bedre orddeling
\usepackage[T1]{fontenc}
\usepackage{lmodern} 
\usepackage{csquotes}


%%Høre til under tabel (Pakker, men skal stå før pgfplot pga. default værdier)%%
\usepackage[table]{xcolor}	

\usepackage{graphicx} 				% Haandtering af eksterne billeder (JPG, PNG, EPS, PDF)
\usepackage{soul}
\usepackage{multirow}               % Fletning af raekker og kolonner (\multicolumn og \multirow)
\usepackage{wrapfig}				%for figure der skal have tekst om sig
\usepackage{float}					% Muliggoer eksakt placering af floats, f.eks. \begin{figure}[H]
\usepackage{enumerate}				% Muliggoer at man kan bruge fx a) i enumerate
\usepackage{pdflscape}				% Muligør landskab på enkelte sider
\usepackage{tabularx}				% Tabeller med X width
\usepackage{mathtools}				% Formler og matematik
\usepackage{pdfpages}				% til at inkludere PDF filer
\usepackage[footnote,draft,english,silent,nomargin]{fixme}	% For at holde styr på mangler i teksten
\usepackage{ifthen}					% If then
\usepackage{hyperref}
\usepackage{textcomp,gensymb}		% symboler til latex
\usepackage{enumitem}				% Muligheder for ref til item
\usepackage{amsfonts}

% ?? Using
%\usepackage[bottom]{footmisc} %fodnoter i bunden af arket

% Complains
% \usepackage{caption}

%--------------------------------------------------------------------
% Billeder bliver lagt i Media
%--------------------------------------------------------------------
\graphicspath{{../figs/}}

%--------------------------------------------------------------------
% Marginer
%--------------------------------------------------------------------
\usepackage{ragged2e,anyfontsize}							%Justering af elementer
\raggedbottom												%Ingen side strech for twopage
\setlrmarginsandblock{3cm}{3cm}{*}							%Højre - venstre
\setulmarginsandblock{3cm}{2.5cm}{*}						%Øverst - nederst
\checkandfixthelayout[nearest]    							%Specifikt valg af højde algoritme
% No more use 2015 \usepackage{fixltx2e}					%Retter forskellige fejl i LaTeX-kernen

%--------------------------------------------------------------------
%  Inholdsfortegnelse
%--------------------------------------------------------------------
\setcounter{tocdepth}{3} % inkludere sub + subsubsection i inholdsfortegnelde
\setsecnumdepth{subsection}


%--------------------------------------------------------------------
% Kapitel formatering
%--------------------------------------------------------------------
\usepackage{titlesec} 

%--------------------------------------------------------------------
% Sektion formatering
%--------------------------------------------------------------------
\titlespacing\section{0pt}
{24pt plus 4pt minus 2pt}{6pt plus 2pt minus 2pt} %halvt mellemrum efter section

\titlespacing\subsection{0pt}
{18pt plus 4pt minus 2pt}{2pt plus 2pt minus 2pt} %kun lidt mellemrum efter subsection

\titlespacing\subsubsection{0pt}
{18pt plus 4pt minus 2pt}{2pt plus 2pt minus 2pt} %kun lidt mellemrum efter subsubsection 

\setlength\parskip{0.2em plus 0.1em}
\setlength\parindent{15pt}



%--------------------------------------------------------------------
% Sidehoved og -fod
%--------------------------------------------------------------------
\let\footruleskip\undefined  %fixer memoir default footruleskip
\usepackage{fancyhdr}
\pagestyle{fancy}
\fancyhf{}
\fancyhead[C]{\textit{<title of thesis>}}
\fancyfoot[RO,LE]{\thepage\ of \thelastpage}  	% sættet sidetal tal h/v efter om der er ulige
												% eller lige sidetal

\fancypagestyle{plain}{% bruges ved Undtagelser						
  \fancyhf{}%
  \fancyfoot[RO,LE]{\thepage\ of \thelastpage}	
  \renewcommand{\headrulewidth}{0pt}			%Ingen linje ved chapter, kun sidetal
}

%--------------------------------------------------------------------
% Literaturliste liste
%--------------------------------------------------------------------
\usepackage{varioref}							% Muliggoer bl.a. krydshenvisninger med sidetal (\vref)
\usepackage{nameref}							% refference der udskriver navn
%\usepackage{natbib}							% Udvidelse med naturvidenskabelige citationsmodeller

%\bibpunct[,]{[}{]}{;}{a}{,}{,} 				% Definerer de 6 parametre ved Harvard henvisning 
												% (bl.a. parantestype og seperatortegn)
%\bibliographystyle{../Latex/Litteratur/plainnat}		% Udseende af litteraturlisten.
%\bibliographystyle{IEEEtranSN}
\usepackage[backend=bibtex,style=ieee,natbib=true,bibencoding=ascii,sorting=none]{biblatex}
\addbibresource{./literature.bib}
%--------------------------------------------------------------------
% Ordliste
%--------------------------------------------------------------------
%\usepackage[nonumberlist,toc]{glossaries}
%\makeglossaries
%\input{./glossary}


%--------------------------------------------------------------------
% Custom funktioner
%--------------------------------------------------------------------

% Forsiden
\usepackage{./frontpage}

%---------------------------CODE--------------------------
\usepackage{listings,color}
%-----------------------------------------------------------
\lstset{
  aboveskip=15pt,
  belowskip=15pt,
  basicstyle=\ttfamily\footnotesize,
  commentstyle=\rm\it,
  flexiblecolumns=false,
  breaklines=true,
  breakautoindent=false
}
\sodef\an{}{0.08em}{0.4em}{0em}

